% Arquivo LaTeX de exemplo de dissertação/tese a ser apresentados à CPG do IME-USP
% 
% Versão 5: Sex Mar  9 18:05:40 BRT 2012
%
% Criação: Jesús P. Mena-Chalco
% Revisão: Fabio Kon e Paulo Feofiloff
%  
% Obs: Leia previamente o texto do arquivo README.txt

\documentclass[11pt,twoside,a4paper]{book}

% ---------------------------------------------------------------------------- %
% Pacotes 
\usepackage[T1]{fontenc}
\usepackage[brazil]{babel}
\usepackage[utf8]{inputenc}
\usepackage[pdftex]{graphicx}           % usamos arquivos pdf/png como figuras
\usepackage{setspace}                   % espaçamento flexível
\usepackage{indentfirst}                % indentação do primeiro parágrafo
\usepackage{makeidx}                    % índice remissivo
\usepackage[nottoc]{tocbibind}          % acrescentamos a bibliografia/indice/conteudo no Table of Contents
\usepackage{courier}                    % usa o Adobe Courier no lugar de Computer Modern Typewriter
\usepackage{type1cm}                    % fontes realmente escaláveis
\usepackage{listings}                   % para formatar código-fonte (ex. em Java)
\usepackage{titletoc}
%\usepackage[bf,small,compact]{titlesec} % cabeçalhos dos títulos: menores e compactos
\usepackage[fixlanguage]{babelbib}
\usepackage[font=small,format=plain,labelfont=bf,up,textfont=it,up]{caption}
\usepackage[usenames,svgnames,dvipsnames]{xcolor}
\usepackage[a4paper,top=2.54cm,bottom=2.0cm,left=2.0cm,right=2.54cm]{geometry} % margens
%\usepackage[pdftex,plainpages=false,pdfpagelabels,pagebackref,colorlinks=true,citecolor=black,linkcolor=black,urlcolor=black,filecolor=black,bookmarksopen=true]{hyperref} % links em preto
\usepackage[pdftex,plainpages=false,pdfpagelabels,pagebackref,colorlinks=true,citecolor=DarkGreen,linkcolor=NavyBlue,urlcolor=DarkRed,filecolor=green,bookmarksopen=true]{hyperref} % links coloridos
\usepackage[all]{hypcap}                    % soluciona o problema com o hyperref e capitulos
\usepackage[round,sort,nonamebreak]{natbib} % citação bibliográfica textual(plainnat-ime.bst)
\usepackage{amsmath}
\usepackage{amsthm}
\usepackage{verbatim}                    % para comentarios em blocos
\fontsize{60}{62}\usefont{OT1}{cmr}{m}{n}{\selectfont}

% ---------------------------------------------------------------------------- %
% Cabeçalhos similares ao TAOCP de Donald E. Knuth
\usepackage{fancyhdr}
\pagestyle{fancy}
\fancyhf{}
\renewcommand{\chaptermark}[1]{\markboth{\MakeUppercase{#1}}{}}
\renewcommand{\sectionmark}[1]{\markright{\MakeUppercase{#1}}{}}
\renewcommand{\headrulewidth}{0pt}

% ---------------------------------------------------------------------------- %
% Redefinindo estilos de teoremas, definições, exemplos, etc.
\theoremstyle{plain}
\newtheorem{thm}{Teorema}[chapter] % redefine a numeração por capítulo

\theoremstyle{definition}
\newtheorem{defn}{Definição}[chapter] % ambiente para definição
\newtheorem{exmp}{Exemplo}[chapter]   % ambiente para exemplos

% ---------------------------------------------------------------------------- %
\graphicspath{{./figuras/}}             % caminho das figuras (recomendável)
\frenchspacing                          % arruma o espaço: id est (i.e.) e exempli gratia (e.g.) 
\urlstyle{same}                         % URL com o mesmo estilo do texto e não mono-spaced
\makeindex                              % para o índice remissivo
\raggedbottom                           % para não permitir espaços extra no texto
\fontsize{60}{62}\usefont{OT1}{cmr}{m}{n}{\selectfont}
\cleardoublepage
\normalsize

% ---------------------------------------------------------------------------- %
% Opções de listing usados para o código fonte
% Ref: http://en.wikibooks.org/wiki/LaTeX/Packages/Listings
\lstset{ %
language=Java,                  % choose the language of the code
basicstyle=\footnotesize,       % the size of the fonts that are used for the code
numbers=left,                   % where to put the line-numbers
numberstyle=\footnotesize,      % the size of the fonts that are used for the line-numbers
stepnumber=1,                   % the step between two line-numbers. If it's 1 each line will be numbered
numbersep=5pt,                  % how far the line-numbers are from the code
showspaces=false,               % show spaces adding particular underscores
showstringspaces=false,         % underline spaces within strings
showtabs=false,                 % show tabs within strings adding particular underscores
frame=single,	                % adds a frame around the code
framerule=0.6pt,
tabsize=2,	                    % sets default tabsize to 2 spaces
captionpos=b,                   % sets the caption-position to bottom
breaklines=true,                % sets automatic line breaking
breakatwhitespace=false,        % sets if automatic breaks should only happen at whitespace
escapeinside={\%*}{*)},         % if you want to add a comment within your code
backgroundcolor=\color[rgb]{1.0,1.0,1.0}, % choose the background color.
rulecolor=\color[rgb]{0.8,0.8,0.8},
extendedchars=true,
xleftmargin=10pt,
xrightmargin=10pt,
framexleftmargin=10pt,
framexrightmargin=10pt
}

% ---------------------------------------------------------------------------- %
% Corpo do texto
\begin{document}
\frontmatter 
% cabeçalho para as páginas das seções anteriores ao capítulo 1 (frontmatter)
\fancyhead[RO]{{\footnotesize\rightmark}\hspace{2em}\thepage}
\setcounter{tocdepth}{2}
\fancyhead[LE]{\thepage\hspace{2em}\footnotesize{\leftmark}}
\fancyhead[RE,LO]{}
\fancyhead[RO]{{\footnotesize\rightmark}\hspace{2em}\thepage}

\onehalfspacing  % espaçamento

% ---------------------------------------------------------------------------- %
% CAPA
% Nota: O título para as dissertações/teses do IME-USP devem caber em um 
% orifício de 10,7cm de largura x 6,0cm de altura que há na capa fornecida pela SPG.
\thispagestyle{empty}
\begin{center}
    \vspace*{2.3cm}
    \textbf{\Large{Título do trabalho a ser apresentado à \\
    CPG para a dissertação/tese}}\\
    
    \vspace*{1.2cm}
    \Large{Rodrigo Augusto Dias Faria}
    
    \vskip 2cm
    \textsc{
    Texto apresentado\\[-0.25cm] 
    ao\\[-0.25cm]
    Instituto de Matemática e Estatística\\[-0.25cm]
    da\\[-0.25cm]
    Universidade de São Paulo\\[-0.25cm]
    para\\[-0.25cm]
    o exame de qualificação\\[-0.25cm]
    do\\[-0.25cm]
    Mestrado em Ciências}
    
    \vskip 1.5cm
    Programa: Ciência da Computação\\
    Orientador: Prof. Dr. Roberto Hirata Junior
    %Coorientador: Prof. Dr. Nome do Coorientador

%   \vskip 1cm
%   \normalsize{Durante o desenvolvimento deste trabalho o autor recebeu auxílio
%   financeiro da CAPES/CNPq/FAPESP}
    
    \vskip 1.5cm % voltar para 0.5cm caso for usar a linha acima
    \normalsize{São Paulo, novembro de 2016}
\end{center}

% ---------------------------------------------------------------------------- %
% Página de rosto (SÓ PARA A VERSÃO DEPOSITADA - ANTES DA DEFESA)
% Resolução CoPGr 5890 (20/12/2010)
%
% IMPORTANTE:
%   Coloque um '%' em todas as linhas
%   desta página antes de compilar a versão
%   final, corrigida, do trabalho
%
%
%\newpage
%\thispagestyle{empty}
%    \begin{center}
%        \vspace*{2.3 cm}
%        \textbf{\Large{Título do trabalho a ser apresentado à \\
%        CPG para a dissertação/tese}}\\
%        \vspace*{2 cm}
%    \end{center}

%    \vskip 2cm

%    \begin{flushright}
%	Esta é a versão original da dissertação/tese elaborada pelo\\
%	candidato (Nome Completo do Aluno), tal como \\
%	submetida à Comissão Julgadora.
 %   \end{flushright}

%\pagebreak


% ---------------------------------------------------------------------------- %
% Página de rosto (SÓ PARA A VERSÃO CORRIGIDA - APÓS DEFESA)
% Resolução CoPGr 5890 (20/12/2010)
%
% Nota: O título para as dissertações/teses do IME-USP devem caber em um 
% orifício de 10,7cm de largura x 6,0cm de altura que há na capa fornecida pela SPG.
%
% IMPORTANTE:
%   Coloque um '%' em todas as linhas desta
%   página antes de compilar a versão do trabalho que será entregue
%   à Comissão Julgadora antes da defesa
%
%
% \newpage
% \thispagestyle{empty}
%     \begin{center}
%         \vspace*{2.3 cm}
%         \textbf{\Large{Título do trabalho a ser apresentado à \\
%         CPG para a dissertação/tese}}\\
%         \vspace*{2 cm}
%     \end{center}

%     \vskip 2cm

%     \begin{flushright}
% 	Esta versão da dissertação/tese contém as correções e alterações sugeridas\\
% 	pela Comissão Julgadora durante a defesa da versão original do trabalho,\\
% 	realizada em 14/12/2010. Uma cópia da versão original está disponível no\\
% 	Instituto de Matemática e Estatística da Universidade de São Paulo.

%     \vskip 2cm

%     \end{flushright}
%     \vskip 4.2cm

%     \begin{quote}
%     \noindent Comissão Julgadora:
    
%     \begin{itemize}
% 		\item Profª. Drª. Nome Completo (orientadora) - IME-USP [sem ponto final]
% 		\item Prof. Dr. Nome Completo - IME-USP [sem ponto final]
% 		\item Prof. Dr. Nome Completo - IMPA [sem ponto final]
%     \end{itemize}
      
%     \end{quote}
% \pagebreak


\pagenumbering{roman}     % começamos a numerar 

% ---------------------------------------------------------------------------- %
% Agradecimentos:
% Se o candidato não quer fazer agradecimentos, deve simplesmente eliminar esta página 
% \chapter*{Agradecimentos}
% Texto texto texto texto texto texto texto texto texto texto texto texto texto
% texto texto texto texto texto texto texto texto texto texto texto texto texto
% texto texto texto texto texto texto texto texto texto texto texto texto texto
% texto texto texto texto. Texto opcional.


% ---------------------------------------------------------------------------- %
% Resumo
\chapter*{{\color{red} Resumo}}

\noindent SOBRENOME, A. B. C. \textbf{Título do trabalho em português}. 
2010. 120 f.
Tese (Doutorado) - Instituto de Matemática e Estatística,
Universidade de São Paulo, São Paulo, 2010.
\\

Elemento obrigatório, constituído de uma sequência de frases concisas e
objetivas, em forma de texto.  Deve apresentar os objetivos, métodos empregados,
resultados e conclusões.  O resumo deve ser redigido em parágrafo único, conter
no máximo 500 palavras e ser seguido dos termos representativos do conteúdo do
trabalho (palavras-chave). 
Texto texto texto texto texto texto texto texto texto texto texto texto texto
texto texto texto texto texto texto texto texto texto texto texto texto texto
texto texto texto texto texto texto texto texto texto texto texto texto texto
texto texto texto texto texto texto texto texto texto texto texto texto texto
texto texto texto texto texto texto texto texto texto texto texto texto texto
texto texto texto texto texto texto texto texto.
Texto texto texto texto texto texto texto texto texto texto texto texto texto
texto texto texto texto texto texto texto texto texto texto texto texto texto
texto texto texto texto texto texto texto texto texto texto texto texto texto
texto texto texto texto texto texto texto texto texto texto texto texto texto
texto texto.
\\

\noindent \textbf{Palavras-chave:} palavra-chave1, palavra-chave2, palavra-chave3.

% ---------------------------------------------------------------------------- %
% Abstract
\chapter*{Abstract}
\noindent SOBRENOME, A. B. C. \textbf{Título do trabalho em inglês}. 
2010. 120 f.
Tese (Doutorado) - Instituto de Matemática e Estatística,
Universidade de São Paulo, São Paulo, 2010.
\\


Elemento obrigatório, elaborado com as mesmas características do resumo em
língua portuguesa. De acordo com o Regimento da Pós- Graduação da USP (Artigo
99), deve ser redigido em inglês para fins de divulgação. 
Text text text text text text text text text text text text text text text text
text text text text text text text text text text text text text text text text
text text text text text text text text text text text text text text text text
text text text text text text text text text text text text.
Text text text text text text text text text text text text text text text text
text text text text text text text text text text text text text text text text
text text text.
\\

\noindent \textbf{Keywords:} keyword1, keyword2, keyword3.

% ---------------------------------------------------------------------------- %
% Sumário
\tableofcontents    % imprime o sumário

% ---------------------------------------------------------------------------- %
\chapter{Lista de Abreviaturas}
\begin{tabular}{ll}
    CIE         & Comissão Internacional de Iluminação (\emph{Commission Internationale de l'Eclairage})\\
    CMY         & Ciano, Magenta e Amarelo (\emph{Cyan, Magenta and Yellow})\\
    HSI         & Matiz, Saturação e Intensidade (\emph{Hue, Saturation, Intensity})\\
    HSL         & Matiz, Saturação e Luminosidade (\emph{Hue, Saturation, Lightness})\\
    HSV         & Matiz, Saturação e Luminância (\emph{Hue, Saturation, Value})\\
    NTSC        & Comitê Nacional do Sistema de Televisão (\emph{National Television System Committee})\\
    PAL         & Linha de Fase Alternada (\emph{Phase Alternation Line})\\
    RGB         & Vermelho, Verde e Azul (\emph{Red, Green and Blue})\\
    SECAM       & Cor Sequencial Com Memória (\emph{Séquentiel Couleur à Mémoire})\\
    UCS         & Escala Uniforme de Cromaticidade (\emph{Uniform Chromaticity Scale})\\
    YIQ         & Luminância, Matiz e Saturação (\emph{Luma, Hue and Saturation})\\
    YUV         & Luminância e Crominância (\emph{Luma and Chrominance})\\
\end{tabular}

% ---------------------------------------------------------------------------- %
\chapter{Lista de Símbolos}
\begin{tabular}{ll}
    $L^*$       & Luminância \\
    $a^*$       & Eixo verde/vermelho no modelo de cores $L^*a^*b^*$ \\
    $b^*$       & Eixo azul/amarelo no modelo de cores $L^*a^*b^*$ \\
    $u^*$       & Eixo verde/vermelho no modelo de cores $L^*u^*v^*$ \\
    $v^*$       & Eixo azul/amarelo no modelo de cores $L^*u^*v^*$ \\
    $\theta$    & Ângulo do matiz no modelo de cores HSI\\
    $max$       & Operador máximo\\
    $min$       & Operador mínimo\\
    $\mu_A(x)$  & Função característica
\end{tabular}

% ---------------------------------------------------------------------------- %
% Listas de figuras e tabelas criadas automaticamente
\listoffigures            
\listoftables            

% ---------------------------------------------------------------------------- %
% Capítulos do trabalho
\mainmatter

% cabeçalho para as páginas de todos os capítulos
\fancyhead[RE,LO]{\thesection}

\singlespacing              % espaçamento simples
%\onehalfspacing            % espaçamento um e meio

\input cap-introducao        % associado ao arquivo: 'cap-introducao.tex'
\input cap-conceitos         % associado ao arquivo: 'cap-conceitos.tex'
\input cap-conclusoes        % associado ao arquivo: 'cap-conclusoes.tex'

% cabeçalho para os apêndices
\renewcommand{\chaptermark}[1]{\markboth{\MakeUppercase{\appendixname\ \thechapter}} {\MakeUppercase{#1}} }
\fancyhead[RE,LO]{}
\appendix

\chapter{Sequências}
\label{ape:sequencias}

Texto texto texto texto texto texto texto texto texto texto texto texto texto
texto texto texto texto texto texto texto texto texto texto texto texto texto
texto texto texto texto texto texto.


\singlespacing

\renewcommand{\arraystretch}{0.85}
\captionsetup{margin=1.0cm}  % correção nas margens dos captions.
%--------------------------------------------------------------------------------------
\begin{table}
\begin{center}
\begin{small}
\begin{tabular}{|c|c|c|c|c|c|c|c|c|c|c|c|c|} 
\hline
\emph{Limiar} & 
\multicolumn{3}{c|}{MGWT} & 
\multicolumn{3}{c|}{AMI} &  
\multicolumn{3}{c|}{\emph{Spectrum} de Fourier} & 
\multicolumn{3}{c|}{Características espectrais} \\
\cline{2-4} \cline{5-7} \cline{8-10} \cline{11-13} & 
\emph{Sn} & \emph{Sp} & \emph{AC} & 
\emph{Sn} & \emph{Sp} & \emph{AC} & 
\emph{Sn} & \emph{Sp} & \emph{AC} & 
\emph{Sn} & \emph{Sp} & \emph{AC}\\ \hline \hline
 1 & 1.00 & 0.16 & 0.08 & 1.00 & 0.16 & 0.08 & 1.00 & 0.16 & 0.08 & 1.00 & 0.16 & 0.08 \\
 2 & 1.00 & 0.16 & 0.09 & 1.00 & 0.16 & 0.09 & 1.00 & 0.16 & 0.09 & 1.00 & 0.16 & 0.09 \\
 2 & 1.00 & 0.16 & 0.10 & 1.00 & 0.16 & 0.10 & 1.00 & 0.16 & 0.10 & 1.00 & 0.16 & 0.10 \\
 4 & 1.00 & 0.16 & 0.10 & 1.00 & 0.16 & 0.10 & 1.00 & 0.16 & 0.10 & 1.00 & 0.16 & 0.10 \\
 5 & 1.00 & 0.16 & 0.11 & 1.00 & 0.16 & 0.11 & 1.00 & 0.16 & 0.11 & 1.00 & 0.16 & 0.11 \\
 6 & 1.00 & 0.16 & 0.12 & 1.00 & 0.16 & 0.12 & 1.00 & 0.16 & 0.12 & 1.00 & 0.16 & 0.12 \\
 7 & 1.00 & 0.17 & 0.12 & 1.00 & 0.17 & 0.12 & 1.00 & 0.17 & 0.12 & 1.00 & 0.17 & 0.13 \\
 8 & 1.00 & 0.17 & 0.13 & 1.00 & 0.17 & 0.13 & 1.00 & 0.17 & 0.13 & 1.00 & 0.17 & 0.13 \\
 9 & 1.00 & 0.17 & 0.14 & 1.00 & 0.17 & 0.14 & 1.00 & 0.17 & 0.14 & 1.00 & 0.17 & 0.14 \\
10 & 1.00 & 0.17 & 0.15 & 1.00 & 0.17 & 0.15 & 1.00 & 0.17 & 0.15 & 1.00 & 0.17 & 0.15 \\
11 & 1.00 & 0.17 & 0.15 & 1.00 & 0.17 & 0.15 & 1.00 & 0.17 & 0.15 & 1.00 & 0.17 & 0.15 \\
12 & 1.00 & 0.18 & 0.16 & 1.00 & 0.18 & 0.16 & 1.00 & 0.18 & 0.16 & 1.00 & 0.18 & 0.16 \\
13 & 1.00 & 0.18 & 0.17 & 1.00 & 0.18 & 0.17 & 1.00 & 0.18 & 0.17 & 1.00 & 0.18 & 0.17 \\
14 & 1.00 & 0.18 & 0.17 & 1.00 & 0.18 & 0.17 & 1.00 & 0.18 & 0.17 & 1.00 & 0.18 & 0.17 \\
15 & 1.00 & 0.18 & 0.18 & 1.00 & 0.18 & 0.18 & 1.00 & 0.18 & 0.18 & 1.00 & 0.18 & 0.18 \\
16 & 1.00 & 0.18 & 0.19 & 1.00 & 0.18 & 0.19 & 1.00 & 0.18 & 0.19 & 1.00 & 0.18 & 0.19 \\
17 & 1.00 & 0.19 & 0.19 & 1.00 & 0.19 & 0.19 & 1.00 & 0.19 & 0.19 & 1.00 & 0.19 & 0.19 \\
17 & 1.00 & 0.19 & 0.20 & 1.00 & 0.19 & 0.20 & 1.00 & 0.19 & 0.20 & 1.00 & 0.19 & 0.20 \\
19 & 1.00 & 0.19 & 0.21 & 1.00 & 0.19 & 0.21 & 1.00 & 0.19 & 0.21 & 1.00 & 0.19 & 0.21 \\
20 & 1.00 & 0.19 & 0.22 & 1.00 & 0.19 & 0.22 & 1.00 & 0.19 & 0.22 & 1.00 & 0.19 & 0.22 \\ \hline 
\end{tabular}
\caption{Exemplo de tabela.}
\label{tab:tab:F5}
\end{small}
\end{center}
\end{table}

      % associado ao arquivo: 'ape-conjuntos.tex'

% ---------------------------------------------------------------------------- %
% Bibliografia
\backmatter \singlespacing   % espaçamento simples
\bibliographystyle{plainnat-ime} % citação bibliográfica textual
\bibliography{bibliografia}  % associado ao arquivo: 'bibliografia.bib'

% ---------------------------------------------------------------------------- %
% Índice remissivo
\index{TBP|see{periodicidade região codificante}}
\index{DSP|see{processamento digital de sinais}}
\index{STFT|see{transformada de Fourier de tempo reduzido}}
\index{DFT|see{transformada discreta de Fourier}}
\index{Fourier!transformada|see{transformada de Fourier}}

\printindex   % imprime o índice remissivo no documento 

\end{document}
