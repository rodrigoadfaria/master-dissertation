%% ------------------------------------------------------------------------- %%
\chapter{Conclusions}
\label{cap:conclusoes}


%------------------------------------------------------
\section{Work plan}
\label{sec:work_plan}

The credits in the required subjects for the master's program in Computer Science at IME-USP were fulfilled from March 2015 until June 2016, according to the table~\ref{tab:subjects}. In the middle of 2015, we started the bibliographic research for the development of this project. During this period up to the examining committee, preliminary experiments were carried out which served as a basis for subsequent steps. It is worth mentioning that the most important recommendation was done by a Committee's member, Prof. Dr. Roberto Marcondes, who has indicated for us the fundamental related paper of this work.


\begin{table}[!htpb]
\centering
\begin{small}
\setlength{\tabcolsep}{6pt}

\begin{tabular}{|c|l|c|}\hline
 \thb{Code} & \thb{Name} & \thb{Conclusion} \\ \hline
 MAC5832 & Machine Learning: Models, Algorithms and Applications    & Jun/2015 \\ \hline
 MAC5744 & Introduction to Computer Graphics                        & Jun/2015 \\ \hline
 MAC5768 & Computer Vision and Image Processing - Part I            & Jun/2015 \\ \hline
 MAC5711 & Analysis of Algorithms                                   & Nov/2015 \\ \hline
 MAC6914 & Learning Methods in Computer Vision                      & Nov/2015 \\ \hline
 MAC4722 & Languages, Automata and Computability                    & Jun/2016 \\ \hline
 MAC5714 & Object-Oriented Programming                              & Jun/2016 \\\hline

\end{tabular}
\end{small}
\caption{Subjects taken in the course of the master's program in Computer Science at IME-USP.}
\label{tab:subjects}
\end{table}

The suggestions and recommendations of the examining committee were considered and reflected in the tasks bellow, represented in the schedule available in Table~\ref{tab:schedule}. We will briefly describe what has happened with each one since then.

\begin{enumerate}
    \item \textbf{Review additional work:} other related work must be added to our project, mainly related to \emph{kernel} methods, which will serve as a basis for understanding and studying the \emph{kernels} technique in \emph{fuzzy} sets proposed by \citet{guevara:14}.

    \item \textbf{Incorporate new datasets:} this task consists of analyzing and integrating new datasets to the project for use in experiments, both of skin color-image datasets as well as others of color images where the classification of interval data can be applied.

    \item \textbf{Investigate appropriate features for interval data classifiers:} the classifiers trained during the preliminary experiments stage consisted only of the color information of the image, regardless of the chosen color space. Therefore, this task aims to understand the role of other attributes, such as texture and local characteristics, during training and whether the inclusion of such attributes, in fact, can lead to performance improvement.

    \item \textbf{Develop new tools:} new tools must be developed to support subsequent experiments, such as the grid search implemented for \emph{FuzzyDT} parameter optimization. All software created by this research project will be made available to the entire scientific community in open source code format.

    \item \textbf{Design new experiments:} based on established tools and datasets, develop and run new experiments.

    \item \textbf{Analysis of results:} the results obtained must be evaluated and reported in the research project and in publications to be carried out. Eventually, fixes on tools or experiments developed will also be aggregated at this stage.

    \item \textbf{Publish results:} scientific papers with the results obtained must be published in magazines or conferences on image processing, classification or other areas related to the context of the research.

    \item \textbf{Write the thesis:} the thesis must be written once previous task have been completed, mainly in relation to the experiments and analysis of results, which will provide consolidated data for the conclusion of the research project.
\end{enumerate}


\begin{table}[!htpb]
\centering

\setlength{\tabcolsep}{6pt} 

\begin{tabular}{|c|c|c|c|c|c|c|c|c|c|c|c|c|}\hline
 & \multicolumn{11}{c|}{Months 2016/2017}\\ \cline{2-12}
\raisebox{1.5ex}{Task} & Dec & Jan & Feb & Mar & Apr & Mai & Jun & Jul & Aug & Sep & Oct \\ \hline

1 & X & X & X &   &   &   &   &   &   &   & \\ \hline
2 & X & X &   &   &   &   &   &   &   &   & \\ \hline
3 &   & X & X &   &   &   &   &   &   &   & \\ \hline
4 &   &   & X & X & X & X &   &   &   &   & \\ \hline
5 &   &   &   &   & X & X & X &   &   &   & \\ \hline
6 &   &   &   &   &   & X & X & X &   &   & \\ \hline
7 &   &   &   &   &   &   & X & X &   &   & \\ \hline
8 &   &   &   &   &   &   &   & X & X & X & X \\ \hline

\end{tabular} 
\caption{Schedule of planned tasks at the time of Examining Committee.}
\label{tab:schedule}
\end{table}

We were trying to apply classification based on interval data for the problem of skin detection until the Examining Committee. That is why the definition of tasks \texttt{\#}1 and \texttt{\#}3. Even though we had to add new papers to be read, they were related to rule-based methods. In addition, task \texttt{\#}2 proved to make sense since three new datasets have been incorporated in the project: Pratheepan, HGR, and Compaq.

One of the methods we explored for data interval classification was the \emph{FuzzyDT} (\emph{Fuzzy Decision Tree}) proposed by~\citet{cintra:13}. As aforementioned, we moved to a different approach, but we also have developed new algorithms, such as the methods described in~\ref{cap:proposed-solution}, as well as new tools to process the datasets, binarize images, and a visualization web application for the problem of skin detection\footnote{Available at \url{https://bitbucket.org/rodrigoadfaria/skin-detector-ws}.}, which is compliant with task \texttt{\#}4. Based on the new methods, we also performed new experiments (task \texttt{\#}5), analyzed the results (task \texttt{\#}6), and published them (task \texttt{\#}7) according to the plan. The latest task (\texttt{\#}8) was to exactly write this thesis.

It is worth noticing that the plan has been shifted ahead for a couple of months. Basically, due the time we had taken to implement the grid search for the parameters tuning and the supplementary neighborhood operations. The conference paper also took some additional time to review, fix recommendations, prepare, and present.


%------------------------------------------------------
\section{Final considerations}
\label{sec:final_considerations}

%------------------------------------------------------
\section{Future work}
\label{sec:future_work}