%% ------------------------------------------------------------------------- %%
\chapter{Plano de Trabalho}
\label{cap:conclusoes}

Os créditos em disciplinas necessários para o programa de mestrado em Ciência da Computação no IME-USP foram cumpridos de Fevereiro de 2015 até Julho de 2016. Em meados de 2015, iniciou-se a pesquisa bibliográfica para o desenvolvimento deste projeto. Durante esse período, até o presente momento, foram realizados os experimentos preliminares que estão incorporados neste texto para a etapa de qualificação. Após apresentação, as recomendações da comissão julgadora serão ponderadas e devem ser refletidas nas atividades subsequentes listadas na seção \ref{sec:atividades_previstas}.

%------------------------------------------------------
\section{Atividades Previstas}
\label{sec:atividades_previstas}
\begin{enumerate}
    \item \textbf{Revisar leituras adicionais:} outras referências deverão ser adicionadas ao presente trabalho, principalmente, conteúdo relacionado a métodos de \emph{kernel}, que servirão de embasamento para compreender e estudar a técnica de \emph{kernels} em conjuntos \emph{fuzzy} proposta por \citet{guevara:14}.

    \item \textbf{Incorporar novos conjuntos de dados:} essa atividade consiste em analisar e integrar novos conjuntos de dados ao projeto para utilização nos experimentos, tanto de bancos de imagens de cor de pele, como outros conjuntos de imagens coloridas onde a classificação de dados intervalares possa ser aplicada.

    \item \textbf{Investigar características passíveis de uso em classificadores de dados intervalares:} os classificadores treinados durante a etapa de experimentos preliminares consistiam apenas da informação de cor da imagem, independente do espaço de cores escolhido. Portanto, essa atividade visa compreender o papel de outros atributos, tais como textura e características locais, durante o treinamento e se a inclusão de tais atributos, de fato, pode acarretar em melhoria de performance.

    \item \textbf{Desenvolver novas ferramentas:} novas ferramentas devem ser desenvolvidas para dar suporte aos experimentos subsequentes, tais como a tabela de busca implementada para otimização de parâmetros do \emph{FuzzyDT}. Todo software criado por este projeto de pesquisa será disponibilizado para toda a comunidade científica em formato de código livre.

    \item \textbf{Elaborar novos experimentos:} com base nas ferramentas desenvolvidas e conjuntos de dados estabelecidos, elaborar e executar novos experimentos.

    \item \textbf{Analisar os resultados:} os resultados obtidos deverão ser avaliados e reportados no projeto de pesquisa e em publicações a serem realizadas. Eventualmente, correções nas ferramentas desenvolvidas e/ou nos experimentos também serão agregadas nesta etapa.

    \item \textbf{Publicar resultados:} artigos científicos com os resultados obtidos devem ser publicados em revistas ou conferências sobre processamento de imagens, classificação ou outras áreas relacionadas com o contexto do projeto de pesquisa.

    \item \textbf{Escrever a dissertação:} a dissertação deverá ser escrita assim que as demais atividades tiverem sido concluídas, principalmente em relação aos experimentos e análise de resultados, o que fornecerá dados consolidados para a conclusão do projeto de pesquisa.
\end{enumerate}

%------------------------------------------------------
\section{Cronograma Proposto}\index{proposto!cronograma}
\label{sec:cronograma}
As atividades listadas na seção \ref{sec:atividades_previstas} serão realizadas de acordo com o cronograma disposto na tabela \ref{tab:cronograma_proposto}. As tarefas foram divididas dentro de um período de onze meses de execução e a previsão de defesa da dissertação é outubro de 2017.

\begin{table}[!htpb]
\centering
% definindo o tamanho da fonte para small
\begin{small} 
  
% redefinindo o espaçamento das colunas
\setlength{\tabcolsep}{6pt} 

% \cline semelhante ao \hline, indicando as colunas com a linha horizontal
% \multicolumn{12}{c|}{Meses} indica que doze colunas serão mescladas
\begin{tabular}{|c|c|c|c|c|c|c|c|c|c|c|c|c|}\hline
 & \multicolumn{11}{c|}{Meses 2016/2017}\\ \cline{2-12}
\raisebox{1.5ex}{Atividade} & Dez & Jan & Fev & Mar & Abr & Mai & Jun & Jul & Ago & Set & Out \\ \hline

1 & X & X & X &   &   &   &   &   &   &   & \\ \hline
2 & X & X &   &   &   &   &   &   &   &   & \\ \hline
3 &   & X & X &   &   &   &   &   &   &   & \\ \hline
4 &   &   & X & X & X & X &   &   &   &   & \\ \hline
5 &   &   &   &   & X & X & X &   &   &   & \\ \hline
6 &   &   &   &   &   & X & X & X &   &   & \\ \hline
7 &   &   &   &   &   &   & X & X &   &   & \\ \hline
8 &   &   &   &   &   &   &   & X & X & X & X \\ \hline

\end{tabular} 
\end{small}
\caption{Cronograma das atividades previstas}
\label{tab:cronograma_proposto}
\end{table}
