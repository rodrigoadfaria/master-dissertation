%% ------------------------------------------------------------------------- %%
\chapter{Plano de Trabalho}
\label{cap:conclusoes}

Os créditos em disciplinas necessários para o programa de mestrado em Ciência da Computação no IME-USP foram cumpridos de Fevereiro de 2015 até Julho de 2016. Em meados de 2015, iniciou-se a pesquisa bibliográfica para o desenvolvimento deste projeto. Durante esse período, até o presente momento, foram realizados os experimentos preliminares que estão incorporados neste texto para a etapa de qualificação. Após apresentação, as recomendações da comissão julgadora serão ponderadas e devem ser refletidas nas atividades subsequentes listadas na seção \ref{sec:atividades_previstas}.

%------------------------------------------------------
\section{Atividades Previstas}
\label{sec:atividades_previstas}
\begin{enumerate}
    \item \textbf{Leituras adicionais:} outras referências devem ser adicionadas ao presente trabalho, principalmente, conteúdo relacionado a métodos de \emph{kernel}, que servirão de embasamento para compreender e estudar a técnica de \emph{kernels} em conjuntos \emph{fuzzy}.
    
    \item \textbf{Análise dos resultados:} os resultados obtidos deverão ser avaliados e reportados no projeto de pesquisa e em publicações a serem realizadas.
    
    \item \textbf{Publicações:} artigos científicos com os resultados obtidos devem ser publicados em revistas ou conferências relacionadas com o contexto do projeto de pesquisa.
    
    \item \textbf{Escrita da dissertação:} a dissertação deve ser escrita assim que as demais atividades tiverem sido concluídas.
\end{enumerate}

%------------------------------------------------------
\section{Cronograma Proposto}\index{proposto!cronograma}
\label{sec:cronograma}
As atividades listadas na seção \ref{sec:atividades_previstas} serão realizados de acordo com o cronograma disposto na tabela \ref{tab:cronograma_proposto}. As tarefas foram divididas dentro de um período de doze meses de execução e a previsão de defesa da dissertação é outubro de 2017.

\begin{table}[!htpb]
\centering
% definindo o tamanho da fonte para small
\begin{small} 
  
% redefinindo o espaçamento das colunas
\setlength{\tabcolsep}{6pt} 

% \cline semelhante ao \hline, indicando as colunas com a linha horizontal
% \multicolumn{12}{c|}{Meses} indica que doze colunas serão mescladas
\begin{tabular}{|c|c|c|c|c|c|c|c|c|c|c|c|c|c|}\hline
 & \multicolumn{12}{c|}{Meses 2016/2017}\\ \cline{2-13}
\raisebox{1.5ex}{Atividade} & Nov & Dez & Jan & Fev & Mar & Abr & Mai & Jun & Jul & Ago & Set & Out \\ \hline

1 & X & X & X &  &  &  &  &  &  &  &  & \\ \hline
2 & X & X & X & X &  &  &  &  &  &  &  & \\ \hline
3 &  & X & X & X & X &  &  &  &  &  &  & \\ \hline
4 &  &  &  & X & X & X & X & X & X &  &  & \\ \hline
5 &  &  &  &  & X & X & X &  &  &  &  & \\ \hline
6 &  &  &  &  & X & X & X & X &  &  &  & \\ \hline
7 &  &  &  &  &  &  & X & X &  &  &  & \\ \hline
8 &  &  &  &  &  &  &  & X & X & X & X & X \\ \hline

\end{tabular} 
\end{small}
\caption{Cronograma das atividades previstas}
\label{tab:cronograma_proposto}
\end{table}
