%% ------------------------------------------------------------------------- %%
\chapter{Conclusions}
\label{cap:conclusoes}

We started this thesis by describing the problem of skin detection, explaining our motivation and determining our main objectives. Further, we reviewed the literature with a large number of works of skin detection, based on color information, comparing their differences in terms of techniques and classifiers, mainly from the point of view of performance, color models, skin color modeling, and datasets. Next, we stated the theoretical concepts that apply to this research where we briefly showed some techniques from image processing and computer vision fields used during the project, such as color space transformation, human skin segmentation, and understanding, etc. Then, we reviewed the method proposed by~\citet{brancati:17} and we added new variations of it.

Also, we showed an implementation of a grid search strategy to figure out the best combination of parameters to the method. Finally, we presented some experimental evaluations of the proposed extensions along with the original method in four widely known datasets: SFA, Pratheepan, HGR, and Compaq.

This chapter address some conclusions regarding our research, objectives, and contributions. In Section~\ref{sec:final_considerations} we sum up the final considerations. Finally, in Section~\ref{sec:future_work}, we present some possible future work.


%------------------------------------------------------
\section{Final considerations}
\label{sec:final_considerations}
The method proposed by~\citet{brancati:17} is a novel rule-based skin detection method that works in the YCbCr color space based on correlation rules that evaluate the combinations of chrominance Cb, Cr values to identify human skin pixels depending on the shape and size of dynamically generated skin color clusters (trapezoids). The method is surprisingly simple and clever and it established a new tier.

Because the authors left the code available, we reproduced the original experiments and also checked if the same color patterns were presented in RGB, HSV, and Lab color spaces, or other applications as finding tree leaves, but the results were not consistent as the original approach for human skin using YCbCr space.

In this research project, we introduced two extensions based on a hypothesis that the original rule could be reversed and also taking into consideration that a human skin pixel does not appear isolated. We also made a third extension that combines the original rule with the reversed one in a more strengthen method in terms of precision. All these extensions are simple and do not hurt the efficiency of the original method.

We tested the extensions in four standard public datasets and the experiments showed that our methods improve the accuracy of skin detection, even when there exists a huge variation in ethnicity and illumination. Moreover, our approach proved to be very competitive, outperforming alternative state-of-the-art work. In addition, we implemented a grid search for the parameters tuning and the supplementary neighborhood operations as well as new tools to process the datasets, binarize images, and a visualization web application for the problem of skin detection\footnote{Available at \url{https://bitbucket.org/rodrigoadfaria/skin-detector-ws}.}.

Our results confirm that skin color is an extremely powerful cue for detecting human skin in unconstrained imagery. Other local properties can be experimented to be used in a future work, along with the methods presented here, such as texture, shape, geometry, and other neighborhood operations.

Finally, part of the work has been published in an international conference  (VISAPP) in earlier 2018.


%------------------------------------------------------
\section{Future work}
\label{sec:future_work}

In the future, extended works could explore further the connectivity of the skin pixels and, because there is so far no explanation why the original method works so well, it would be valuable to statistically analyze the shape of the trapezoids on the YCbCr space and try to correlate with the classification accuracy.

Our intuition, based on the experimental results, says that trapezoids features such as size, area, symmetry, and others, could be used to establish a relationship with the classification accuracy. Moreover, the shape of the trapezoids could be previously processed, for instance by filtering image illumination, to obtain better classification results.

The neighborhood operations add some additional computational cost but the implementation can be enhanced as well as other techniques applied, such as applying weights when counting neighboring pixels.

We have also implemented a grid search algorithm to understand the parameters selection in Section~\ref{sec:grid_search_experiments}. This approach can be improved and we could try to learn the best parameters of the trapezoids by means of Machine Learning techniques.

A recent test paper comparing different color spaces suggests that YCgCr is also a good candidate for this problem~\citep{chaves:10}. In fact, YCbCr and YCgCr color spaces are very similar. Therefore, the same color relations can be presented in this space.

Another important investigation that could be addressed is related to the ranges of the chromaticity (Cr, Cb) channels used in the definition of the correlation rules. In~\citet{brancati:17}, as well as in our proposed methods, the thresholds used were given by~\citet{chai:99}. However,~\citet{basilio:11} found out that the thresholds were robust only against images with Caucasian people. In the end,~\citet{basilio:11} proposed different values that seems to be more robust to find human skin from different people races. This simple change may also result in improved human skin segmentation quality.