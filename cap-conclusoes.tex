%% ------------------------------------------------------------------------- %%
\chapter{Conclusões}
\label{cap:conclusoes}

Texto texto texto texto texto texto texto texto texto texto texto texto texto
texto texto texto texto texto texto texto texto texto texto texto texto texto
texto texto texto texto texto texto\footnote{Exemplo de referência para página
Web: \url{www.vision.ime.usp.br/~jmena/stuff/tese-exemplo}}.

%------------------------------------------------------
\section{Considerações Finais} 

Texto texto texto texto texto texto texto texto texto texto texto texto texto
texto texto texto texto texto texto texto texto texto texto texto texto texto
texto texto texto texto texto texto. 

%------------------------------------------------------
\section{Sugestões para Pesquisas Futuras} 

Texto texto texto texto texto texto texto texto texto texto texto texto texto
texto texto texto texto texto texto texto texto texto texto texto texto texto
texto texto texto texto texto texto.

Finalmente, leia o trabalho de \citet{alon09:how} no qual apresenta-se
uma reflexão sobre a utilização da Lei de Pareto para tentar definir/escolher
problemas para as diferentes fases da vida acadêmica.  A direção dos novos
passos para a continuidade da vida acadêmica deveriam ser discutidos com seu
orientador.

%------------------------------------------------------
\section{Cronograma Proposto}
A realização de todos os experimentos propostos, desenvolvimento das ferramentas, análise dos resultados e a elaboração da dissertação, serão realizados de acordo com o cronograma disposto na tabela \ref{tab:cronograma_proposto}. As tarefas foram divididas dentro de um período de doze meses de execução e a previsão de defesa da dissertação é outubro de 2017.

\begin{table}[!htpb]
\centering
% definindo o tamanho da fonte para small
\begin{small} 
  
% redefinindo o espaçamento das colunas
\setlength{\tabcolsep}{6pt} 

% \cline semelhante ao \hline, indicando as colunas com a linha horizontal
% \multicolumn{12}{c|}{Meses} indica que doze colunas serão mescladas
\begin{tabular}{|c|c|c|c|c|c|c|c|c|c|c|c|c|c|}\hline
 & \multicolumn{12}{c|}{Meses 2016/2017}\\ \cline{2-13}
\raisebox{1.5ex}{Atividade} & Nov & Dez & Jan & Fev & Mar & Abr & Mai & Jun & Jul & Ago & Set & Out \\ \hline

1 & X & X & X &  &  &  &  &  &  &  &  & \\ \hline
2 & X & X & X & X &  &  &  &  &  &  &  & \\ \hline
3 &  & X & X & X & X &  &  &  &  &  &  & \\ \hline
4 &  &  &  & X & X & X & X & X & X &  &  & \\ \hline
5 &  &  &  &  & X & X & X &  &  &  &  & \\ \hline
6 &  &  &  &  & X & X & X & X &  &  &  & \\ \hline
7 &  &  &  &  &  &  & X & X &  &  &  & \\ \hline
8 &  &  &  &  &  &  &  & X & X & X & X & X \\ \hline

\end{tabular} 
\end{small}
\caption{Cronograma das atividades previstas}
\label{tab:cronograma_proposto}
\end{table}
\index{proposto!cronograma|)}
