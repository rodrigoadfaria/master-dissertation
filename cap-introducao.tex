%% ------------------------------------------------------------------------- %%
\chapter{Introdução}
\label{cap:introducao}


%% ------------------------------------------------------------------------- %%
\section{Motivação}
\label{sec:consideracoes_preliminares}

Há inúmeros indivíduos ou observações na natureza que não podem ser classificados com conjuntos clássicos pelo fato de que a relação de pertinência não é bem definida \citep{pedrycz:98}. Especificamente no campo da visão computacional, vários problemas reais de classificação só podem ser resolvidos quando da análise do contexto onde o problema está inserido.

Sendo assim, a motivação deste trabalho está em analisar problemas reais de visão computacional cujos conjuntos de dados possuem incerteza e imprecisão, modelando-os por meio de conjuntos \emph{fuzzy}, explorando a capacidade desses conjuntos de expressarem transições graduais de pertinência e não pertinência.


%% ------------------------------------------------------------------------- %%
\section{Trabalhos relacionados}
\label{sec:trabalhos_relacionados}

Em alguns problemas de classificação, o uso de dados intervalares pode ser uma ferramenta poderosa na obtenção de bons resultados. Intervalos representados por conjuntos \emph{fuzzy} também têm grande valia nesta tarefa. Com base nessa característica, \citet{umano:94} propôs um novo algoritmo para gerar uma árvore de decisão \emph{fuzzy}, a partir de dados numéricos, usando conjuntos \emph{fuzzy} previamente definidos. Essa implementação é uma adaptação do algoritmo Divisor Iterativo 3 (ID3) clássico proposto por \citet{quinlan:86}, brevemente discutido na seção \ref{sec:clasificadores_arvores_decisao}. A diferença fundamental para o ID3 está no método de seleção do atributo que particiona o conjunto de dados. Assim como no ID3, aqui também é utilizado o ganho de informação como critério, porém, ao invés de aplicar o valor dos atributos no cálculo, o algoritmo computa pelo grau de pertinência dos conjuntos \emph{fuzzy}. Outra propriedade particular da árvore \emph{fuzzy} está no processo de inferência, pois permite que mais de um ramo seja ativado quando um atributo é testado na avaliação do grau de pertinência.

O \emph{fuzzy} ID3 foi utilizado por \citet{bhatt:09} para segmentação de regiões de pele. O conjunto de dados aplicado no experimento contém os três canais de cores do modelo Vermelho, Verde e Azul (RGB) e também foi utilizado nos experimentos deste trabalho, conforme pode ser visto no Capítulo \ref{cap:experimentos}. Antes da indução da árvore, os dados de treinamento foram aglomerados em cinco \emph{clusters} por meio do algoritmo \emph{fuzzy c-means}. Os \emph{clusters} foram aproximados com função de pertinência gaussiana. A taxa de erro obtida foi de 94,1\%.

Outra variante de árvore de decisão para conjuntos \emph{fuzzy} é a proposta de \citet{cintra:13}, porém, baseada na versão clássica C4.5, uma versão estendida do ID3 também criada por \citet{quinlan:93}. Chamado de \emph{FuzzyDT}, o algoritmo igualmente usa a entropia e o ganho de informação do C4.5 como medidas para decidir sobre a importância dos atributos. A estratégia de indução da árvore também é a mesma, ou seja, os dados são particionados recursivamente criando ramos até que uma classe é atribuída a cada folha. Porém, antes da indução da árvore, os atributos contínuos são definidos em termos de conjuntos \emph{fuzzy}. Em outras palavras, a árvore é induzida com os valores dos atributos "discretizados". O domínio dos conjuntos \emph{fuzzy} devem ser estabelecidos previamente pelo usuário. Experimentos foram realizados com 16 conjuntos de dados distintos obtidos de \citet{lichman:13}, comparando o desempenho do algoritmo com o C4.5 em termos da taxa de erro, número médio de regras e condições geradas por ambas técnicas. Em resumo, o \emph{FuzzyDT} produziu resultados satisfatórios na maioria dos conjuntos de dados avaliados. O algoritmo foi utilizado para experimentos preliminares que podem ser vistos no Capítulo \ref{cap:experimentos}.

\citet{guevara:14} propõe uma formulação geral de \emph{kernel} sobre conjuntos \emph{fuzzy} e, além disso, define exemplos de construção de \emph{kernels fuzzy} positivos definidos, tais como, polinomial e gaussiano, para resolver problemas de aprendizagem computacional, em particular de aprendizagem supervisionada, como os estudos presentes neste trabalho.


%% ------------------------------------------------------------------------- %%
\section{Objetivos}
\label{sec:objetivo}

O principal objetivo deste projeto de mestrado está na avaliação de conjuntos \emph{fuzzy} na aplicação em problemas de classificação em visão computacional, estabelecendo mecanismos comparativos, por meio de indicadores quantitativos e qualitativos, para mensurar seu desempenho em relação aos conjuntos clássicos.

Além disso, a escolha do espaço de cores para a modelagem dos dados como conjuntos \emph{fuzzy} também será analisada com o intuito de compreender sua influência nos resultados obtidos pelo classificador.


%% ------------------------------------------------------------------------- %%
\section{Organização do texto}
\label{sec:organizacao_trabalho}

Quanto à organização deste trabalho, no Capítulo~\ref{cap:conceitos} são explicitados conceitos fundamentais sobre diferentes modelos de cores, teoria \emph{fuzzy} e classificadores utilizados no desenvolvimento deste estudo. O Capítulo~\ref{cap:experimentos}, são expostos os resultados de experimentos preliminares com conjuntos de dados de pele de seres humanos, bem como a avaliação da influência do modelo de cores escolhido para tratar este tipo de problema de classificação. Por fim, o Capítulo~\ref{cap:conclusoes} apresenta o plano de trabalho com as atividades a serem realizadas no âmbito desta pesquisa.
