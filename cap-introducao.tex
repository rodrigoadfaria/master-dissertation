%% ------------------------------------------------------------------------- %%
\chapter{Introdução}
\label{cap:introducao}


%% ------------------------------------------------------------------------- %%
\section{Motivação}
\label{sec:consideracoes_preliminares}

Há inúmeros indivíduos ou observações na natureza que não podem ser classificados com conjuntos clássicos pelo fato de que a relação de pertinência não é bem definida \citep{pedrycz:98}. Especificamente no campo da visão computacional, vários problemas reais de classificação só podem ser resolvidos quando da análise do contexto onde o problema está inserido.

Sendo assim, a motivação deste trabalho está em analisar problemas reais de visão computacional cujos conjuntos de dados possuem incerteza e imprecisão, modelando-os por meio de conjuntos \emph{fuzzy}, explorando a capacidade desses conjuntos de expressarem transições graduais de pertinência e não pertinência.
 

%% ------------------------------------------------------------------------- %%
\section{Trabalhos Relacionados}
\label{sec:trabalhos_relacionados}

A referência fundamental desta pesquisa é o trabalho de \citet{guevara:14} que propõe uma formulação geral de \emph{kernel} sobre conjuntos \emph{fuzzy} e, além disso, define exemplos de construção de \emph{kernels fuzzy} positivos definidos, tais como, polinomial e gaussiano, para resolver problemas de aprendizagem computacional, em particular de aprendizagem supervisionada, como os estudos presentes neste trabalho.

\citet{umano:94} propôs um novo algoritmo para gerar uma árvore de decisão \emph{fuzzy} a partir de dados numéricos usando conjuntos \emph{fuzzy} previamente definidos. Essa implementação é uma extensão do algoritmo ID3 clássico proposto por \citet{quinlan:86}. Outra variante de árvore de decisão para conjuntos \emph{fuzzy} é a proposta de \citet{cintra:13}, porém, baseada na versão clássica C4.5, também criada por \citet{quinlan:96}.


%% ------------------------------------------------------------------------- %%
\section{Objetivos}
\label{sec:objetivo}

O principal objetivo deste projeto de mestrado está na avaliação de conjuntos \emph{fuzzy} na aplicação em problemas de classificação em visão computacional, estabelecendo mecanismos comparativos, por meio de indicadores quantitativos e qualitativos, para mensurar seu desempenho em relação aos conjuntos clássicos.

Além disso, a escolha do espaço de cores para a modelagem dos dados como conjuntos \emph{fuzzy} também será analisada com o intuito de compreender sua influência nos resultados obtidos pelo classificador.


%% ------------------------------------------------------------------------- %%
\section{Organização do Texto}
\label{sec:organizacao_trabalho}

Quanto à organização deste trabalho, no Capítulo~\ref{cap:conceitos} são explicitados conceitos fundamentais sobre diferentes modelos de cores, teoria \emph{fuzzy} e classificadores utilizados no desenvolvimento deste estudo. O Capítulo~\ref{cap:experimentos}, são expostos os resultados de experimentos preliminares com conjuntos de dados de pele de seres humanos, bem como a avaliação da influência do modelo de cores escolhido para tratar este tipo de problema de classificação. Por fim, o Capítulo~\ref{cap:conclusoes} apresenta o plano de trabalho com as atividades a serem realizadas no âmbito desta pesquisa.
