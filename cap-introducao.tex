%% ------------------------------------------------------------------------- %%
\chapter{Introduction}
\label{cap:introducao}
Skin detection can be defined as the process of identifying skin-colored pixels in an image. It plays an important role in a wide range of image processing and computer vision applications such as face detection, pornographic image filtering, gesture analysis, face tracking, video surveillance systems, medical image analysis, and other human-related image processing applications.

The problem is complex because of the numerous similar materials with human skin tone and texture, and also because of illumination conditions, ethnicity, sensor capturing singularities, geometric variations, etc. Because it is a primary task in image processing, additional requirements as real time processing, robustness and accuracy are also desirable.

Skin color is a strong characteristic and it is used in most algorithms for skin detection. It is normally used along with other features such as shape, texture, and geometry, or even as a preliminary step to classify regions of interest in an image.

The human skin color pixels have a restricted range of hues and are not deeply saturated, since the appearance of skin is formed by a combination of blood (red) and melanin (brown, yellow), which leads the human skin color to be clustered within a small area in the color space~\citep{fleck:96}.

The choice of a color space is also a key point of a feature-based method when using skin color as a detection cue. Due to its sensitivity to illumination, the input image is, in general, first transformed into a color space whose luminance and chrominance components can be separate to mitigate the problem~\citep{vezhnevets:03}.

Color has the ability of functioning as a descriptor that often simplifies the identification and extraction of an object in a scene. Moreover, the ability of humans to discern thousands of tonalities and intensities compared to only a few dozen levels of gray, put the color as a strong candidate feature in computer vision and image processing applications~\citep{gonzalez:02}.

Basically, there are three approaches for skin detection: rule-based, machine learning based and hybrid. They differ in terms of classification accuracy and computational efficiency. Machine learning and hybrid methods require a training set, from which the decision rules are learned. Such approaches outperform the rule-based methods but require a large and representative training dataset as well as it takes a long classification time, which can be a deal breaker for real time applications~\citep{kakumanu:07}.

In this work we propose an improvement of a novel method on rule-based skin detection that works in the YCbCr color space~\citep{brancati:17}. Our motivation is based on the hypothesis that the original rule can be complemented with another rule that is a reversal interpretation of the one proposed originally. Besides that, we also take in consideration that a skin pixel does not appear isolated, so we propose another variation based on neighborhood operations. The set of rules evaluate the combinations of chrominance Cb, Cr values to identify the skin pixels depending on the shape and size of dynamically generated skin color clusters \citep{brancati:17}. The method is very efficient in terms of computational effort as well as robust in very complex image scenes.

%% ------------------------------------------------------------------------- %%
\section{Motivation}
\label{sec:motivation}

Perhaps this was the most iconic section I have written in this thesis. And here, first of all, I would kindly like to ask my dear Professor and advisor Roberto Hirata to excuse me for (first) summarizing the reasons that brought me to this Master's program, from a personal point of view, of course. Later, I will write down the specific reasons to choose this research topic.

When I look at the word motivation, it is impossible not to establish self-questioning such as: (i) What made you do a master's degree in Computer Science? (ii) Why did not you pursue some other type of postgraduate course? And, of course, (iii) why did you choose this topic for the research? Responding to these questions, I must possibly bring to the reader the feeling that has moved me to this point.

I was born into a family of five siblings (seven in total) and grew up in a very poor rural zone of the interior of Minas Gerais. I always believed that studying was the only way to change the future of my own and my family. I graduated from public school and became a bachelor in Information Systems with a scholarship. However, I always had the dream of studying at a renowned teaching institution such as the University of São Paulo. Partly because I would like to become a Teacher, something I did for a few opportunities. That answers the first question.

I am really passionate about my profession. I have ten years of experience in software development, embedded systems and, more recently, in the semiconductor industry. I felt that something was missing in my background education. The master's degree showed me, mainly, the magic of formal mathematics and the importance it has in fundamental areas in society such as computing, engineering, physics and others. And the Institute of Mathematics and Statistics of USP played a decisive role in this respect, which answers the second question.

And last but not least, the subject of the research (third question) has been something that was attractive for us from the beginning of the program. First, with a project on race classification in partnership with the industry. Then, with the intensification in the search for related works, with the problem of skin detection. The latter led us to the brilliant work of~\citet{brancati:17}: a new rule-based skin detection method that works in the YCbCr color space. Here, more specifically, our motivation was to propose improvements based on the hypothesis that: (1) the original rule can be reversed and, (2) human skin pixels do not appear isolated, i.e. neighborhood operations are taken in consideration.


%% ------------------------------------------------------------------------- %%
\section{Contributions}
\label{sec:contributions}

In this work, we have done a comprehensive and detailed study of various methods of skin detection within those based on rules. On the basis of~\citet{brancati:17}, seen by us as the state of the art in this field, we created variations that brought significant improvements. In addition, we analyzed the methods in question in order to provide the researchers, practitioners, enthusiasts, and other readers with a detailed understanding of the nuances involved in those methods, such as parameter selection and optimization, through a series of quantitative experiments, as well as qualitative analysis based on our observations.

Thus, we can enumerate some of the contributions that came as a result of this research project:
\begin{enumerate}
    \item Implementation of three variations of the skin detection method proposed by~\citet{brancati:17} on the basis of the inversely proportional behavior of the chrominance components ($Cb$ and $Cr$) of the $YCbCr$ color model. Furthermore, extensive quantitative and qualitative experiments performed in a wide range of image datasets well-known in this field;
    \item Adapted version of the neighborhood method (8-$neighbors$ window) presented in section~\ref{sec:neighborhood_extended_method};
    \item A grid search implementation to try different combinations of trapezoids parameters in order to optimize them (if possible) and understand those who have been used so far.
\end{enumerate}

Part of our contributions was published earlier in 2018 in the \emph{Proceedings of the 13th International Joint Conference on Computer Vision, Imaging and Computer Graphics Theory and Applications - VISAPP}~\citep{faria:18}.


%% ------------------------------------------------------------------------- %%
\section{Objectives}
\label{sec:objectives}

Skin detection is a very complex problem due the numerous similar materials with human skin tone and texture, and also because of illumination conditions, ethnicity, sensor capturing singularities, geometric variations, etc. Because it is a primary task in image processing, additional requirements as real time processing, robustness and accuracy are also desirable.

Although many advances have been observed in literature, from what we have seen until the development of this research, it is possible to say that it is not yet a completely solved problem. Basically, there are three approaches for skin detection: rule-based, machine learning based and hybrid. They differ in terms of classification accuracy and computational efficiency. In general, rule-based methods do not require a training step and they can be very competitive in terms of computational cost.

Therefore, the main objective of this research is to create or improve new methods for skin detection, according to the rule-based approach. More specifically, our main objective is to achieve improvements in the method proposed by~\citet{brancati:17} in order to reduce the false positive rate. In addition, we have other secondary objectives, such as:
\begin{itemize}
    \item Publish articles with the results in renowned conference or journals in the area;
    \item Complete the master's degree in three years, which includes the credits in disciplines as well as the research project and this thesis.
\end{itemize}



%% ------------------------------------------------------------------------- %%
\section{Organization}
\label{sec:text_organization}
In this chapter we have presented the background of this work as well as the motivation, main contributions, and objectives behind it. Chapter~\ref{cap:related-work} presents other relevant research works that also address the problem of skin detection in several approaches. In Chapter~\ref{cap:conceitos} we provide an overview of the theoretical concepts that apply to this research. Next, in Chapter~\ref{cap:proposed-solution} we present a state of the art skin detection method recently developed by~\citet{brancati:17}. We review the method and extend it adding more rules to enforce the constraints and seeking for a better accuracy in terms of false positive rate without hurting the performance of the original method. Then, in Chapter~\ref{cap:experimentos} we present the evaluation of the proposed extensions along with the original method in four widely known datasets: SFA, Pratheepan, HGR, and Compaq. In addition, a
brief definition of the evaluation metrics used is shown for the sake of clarity. Finally, Chapter~\ref{cap:conclusoes} winds up this thesis by discussing our observations along the research, focused on the experimental results, and directs the readers towards future works.